% since 1995    
\urldef\pubmedquery\url{https://www.ncbi.nlm.nih.gov/pubmed?term=(%221995%22%5BDate%20-%20Publication%5D%20%3A%20%222017%22%5BDate%20-%20Publication%5D)}

% since 1900
\urldef\pubmedquery\url{https://www.ncbi.nlm.nih.gov/pubmed/?term=%221900%22%5BPDAT%5D%20%3A%20%222017%22%5BPDAT%5D&cmd=DetailsSearch}

\usepackage{multirow}
\usepackage{subfigure}
\usepackage{pbox}
\usepackage{bm}
\usepackage{latexsym}
\usepackage{csquotes}
\usepackage{amsmath}
\usepackage{pdflscape}
\usepackage{fancyref}

% for tikz compatibility
%\noautomath
\usepackage{varwidth}
\usepackage{tikz}
\usetikzlibrary{automata, calc, backgrounds, er, trees, fit, positioning, arrows, chains, shapes.geometric, decorations.pathreplacing, decorations.pathmorphing,shapes, matrix, shapes.symbols}
\usepackage{tikz-qtree}
\usepackage{tikz-dependency}
%\usepackage{gb4e}
\usepackage{enumitem}
%\usepackage{subcaption}
%\usepackage[lined]{algorithm2e}

%\bibliographystyle{plainnat}
% \usepackage[american]{babel}
% \usepackage[
%         backend=biber,
%         style=authoryear,
%         sorting=nyt,
%         natbib=true,
%         doi=true,
%         arxiv=abs,
%         url=true
% ]{biblatex}

% \addbibresource{dissertation.bib}
% \DefineBibliographyStrings{american}{phdthesis = {PhD dissertation}}
% \preto\fullcite{\AtNextCite{\defcounter{maxnames}{99}}}


\usepackage{placeins}

\let\Oldsection\section
\renewcommand{\section}{\FloatBarrier\Oldsection}

\let\Oldsubsection\subsection
\renewcommand{\subsection}{\FloatBarrier\Oldsubsection}

\let\Oldsubsubsection\subsubsection
\renewcommand{\subsubsection}{\FloatBarrier\Oldsubsubsection}


\usepackage[framemethod=TikZ]{mdframed}
\mdfdefinestyle{proceduredesc}{%
    linecolor=black,
    outerlinewidth=0.5pt,
    roundcorner=5pt,
    innertopmargin=\baselineskip,
    innerbottommargin=\baselineskip,
    innerrightmargin=15pt,
    innerleftmargin=15pt,
    backgroundcolor=gray!10!white}


\usepackage{epigraph}
\renewcommand{\epigraphsize}{\small}

\setlength{\epigraphwidth}{1.0\textwidth}

\renewcommand{\textflush}{flushright} \renewcommand{\sourceflush}{flushright}

\let\originalepigraph\epigraph 
\renewcommand\epigraph[2]{\originalepigraph{\textit{#1}}{\textsc{#2}}}

\newcommand{\reach}{Reach}
\newcommand{\odin}{Odin}
\newcommand{\grayrule}{}

% define extractor language
\lstdefinelanguage{yaml}{% alsoletter={@=><&?.*+^,_/|[]},
	morestring=[b]", keywords={true, false, null, name, action, example,
		priority, type, label, unit, pattern}, keywordstyle=\textbf,
	ndkeywords={trigger, ?<trigger>, theme, theme?, ?<theme>, @theme,
		cause, cause?, ?<cause>, @cause, site, site?, @site, product,
		product?, @site, controller, controlled, @controlled,
		?<witness>, @witness,
		?<entity>, @entity, entity, word, lemma, tag,
		?<location>, @location, location,
		?<robber>, @robber, robber,
		@property, property,
		agent, vehicle,
		obstacle,
		title,
		@person, person,
		@num,
		dancer, partner,
		city},
	ndkeywordstyle=\color{arsenic}\bfseries, comment=[l]{\#},
	commentstyle=\color{coolgrey}\textit, stringstyle=\ttfamily,
	sensitive=true }

\lstdefinestyle{yaml-style}{%
	language=yaml,
	basicstyle=\ttfamily\footnotesize,
	%basewidth={0.5em,0.5em},
	xleftmargin=9pt,
	xrightmargin=-8pt,
	%xleftmargin=20pt,
	%xrightmargin=-8pt,
	framexleftmargin=16pt,
	framexrightmargin=-10pt,
	%framextopmargin=2pt,
	%framexbottommargin=0pt,
	%frame=trBL,
	%frameround=fttt,
	frame=tb,
	captionpos=t,
	backgroundcolor=\color{anti-flashwhite},
	rulecolor=\color{black},
	extendedchars=true,
	showstringspaces=false,
	showspaces=false,
	numbers=left,
	numberstyle=\tiny,
	numbersep=8pt,
	stepnumber=1,
	tabsize=2,
	breaklines=true,
	showtabs=false,
	escapeinside={(*@}{@*)}
}

\lstnewenvironment{yaml}[1]{%
	\lstset{style=yaml-style}}{%
	\captionof{lstlisting}{#1}
}

%\renewcommand{\lstlistingname}{Rule}% Listing -> Example
\renewcommand{\lstlistingname}{\bfseries Rule}
\makeatletter
\def\fnum@lstlisting{%
  \lstlistingname
  \ifx\lst@@caption\@empty\else~\thelstlisting\normalfont\fi}%
\makeatother

%%%%%%%%%%%%%%%%%%%%%%%%%%%
% custom figures
%%%%%%%%%%%%%%%%%%%%%%%%%%%
\usepackage{float}
\newfloat{example}{thp}{lop}
\floatname{example}{Example}

%%%%%%%%%%%%%%%%%%%%%%%%%%%%
% define colors
\definecolor{lightgray}{gray}{0.7}
\definecolor{anti-flashwhite}{rgb}{0.95, 0.95, 0.96}
\definecolor{coolgrey}{rgb}{0.55, 0.57, 0.67}
\definecolor{arsenic}{rgb}{0.23, 0.27, 0.29}
\definecolor{purpleish}{cmyk}{0.75,0.75,0,0}
\definecolor{light-gray}{gray}{0.9}
\definecolor{light-red}{rgb}{1,.8,.8}
\definecolor{pinegreen}{rgb}{0.0, 0.47, 0.44}
%%%%%%%%%%%%%%%%%%%%%%%%%%%%

\DeclareMathOperator*{\argmin}{argmin}
\DeclareMathOperator*{\argmax}{argmax}

%\newcommand{\todo}[1]{\textcolor{red}{TODO: #1}}
\newcolumntype{x}[1]{>{\centering\arraybackslash\hspace{0pt}}p{#1}}
\definecolor{tbl}{rgb}{.85,.95,.85}

% Render DOIs as URLs
\renewcommand{\doi}[1]{\href{http://dx.doi.org/#1}{\textsc{doi}: #1}}
 
% fix to avoid weird interaction with Tufte
\usepackage{gb4e}
\noautomath
%\usepackage{linguex}

% Add the following code AFTER the gb4e package has been loaded.
% This will restore the Tufte-LaTeX definition of footnotes (as sidenotes)
% but adds in the two lines of code used by the gb4e package.
\makeatletter
\renewcommand\@footnotetext[2][0pt]{%
  \@noftnotefalse\setcounter{fnx}{0}% added by gb4e
  \marginpar{%
    \hbox{}\vspace*{#1}%
    \def\baselinestretch {\setspace@singlespace}%
    \reset@font\footnotesize%
    \@tufte@margin@par% use parindent and parskip settings for marginal text
    \vspace*{-1\baselineskip}\noindent%
    \protected@edef\@currentlabel{%
       \csname p@footnote\endcsname\@thefnmark%
    }%
    \color@begingroup%
       \@makefntext{%
         \ignorespaces#2%
       }%
    \color@endgroup%
  }%
  \@noftnotetrue% added by g4be
}%
\makeatother
% End of Tufte-LaTeX-related code.